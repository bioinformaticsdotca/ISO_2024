% Options for packages loaded elsewhere
\PassOptionsToPackage{unicode}{hyperref}
\PassOptionsToPackage{hyphens}{url}
\documentclass[
]{book}
\usepackage{xcolor}
\usepackage{amsmath,amssymb}
\setcounter{secnumdepth}{5}
\usepackage{iftex}
\ifPDFTeX
  \usepackage[T1]{fontenc}
  \usepackage[utf8]{inputenc}
  \usepackage{textcomp} % provide euro and other symbols
\else % if luatex or xetex
  \usepackage{unicode-math} % this also loads fontspec
  \defaultfontfeatures{Scale=MatchLowercase}
  \defaultfontfeatures[\rmfamily]{Ligatures=TeX,Scale=1}
\fi
\usepackage{lmodern}
\ifPDFTeX\else
  % xetex/luatex font selection
\fi
% Use upquote if available, for straight quotes in verbatim environments
\IfFileExists{upquote.sty}{\usepackage{upquote}}{}
\IfFileExists{microtype.sty}{% use microtype if available
  \usepackage[]{microtype}
  \UseMicrotypeSet[protrusion]{basicmath} % disable protrusion for tt fonts
}{}
\makeatletter
\@ifundefined{KOMAClassName}{% if non-KOMA class
  \IfFileExists{parskip.sty}{%
    \usepackage{parskip}
  }{% else
    \setlength{\parindent}{0pt}
    \setlength{\parskip}{6pt plus 2pt minus 1pt}}
}{% if KOMA class
  \KOMAoptions{parskip=half}}
\makeatother
\usepackage{longtable,booktabs,array}
\usepackage{calc} % for calculating minipage widths
% Correct order of tables after \paragraph or \subparagraph
\usepackage{etoolbox}
\makeatletter
\patchcmd\longtable{\par}{\if@noskipsec\mbox{}\fi\par}{}{}
\makeatother
% Allow footnotes in longtable head/foot
\IfFileExists{footnotehyper.sty}{\usepackage{footnotehyper}}{\usepackage{footnote}}
\makesavenoteenv{longtable}
\usepackage{graphicx}
\makeatletter
\newsavebox\pandoc@box
\newcommand*\pandocbounded[1]{% scales image to fit in text height/width
  \sbox\pandoc@box{#1}%
  \Gscale@div\@tempa{\textheight}{\dimexpr\ht\pandoc@box+\dp\pandoc@box\relax}%
  \Gscale@div\@tempb{\linewidth}{\wd\pandoc@box}%
  \ifdim\@tempb\p@<\@tempa\p@\let\@tempa\@tempb\fi% select the smaller of both
  \ifdim\@tempa\p@<\p@\scalebox{\@tempa}{\usebox\pandoc@box}%
  \else\usebox{\pandoc@box}%
  \fi%
}
% Set default figure placement to htbp
\def\fps@figure{htbp}
\makeatother
\setlength{\emergencystretch}{3em} % prevent overfull lines
\providecommand{\tightlist}{%
  \setlength{\itemsep}{0pt}\setlength{\parskip}{0pt}}
\usepackage[]{natbib}
\bibliographystyle{plainnat}
\usepackage{booktabs}

\usepackage{color}
\usepackage{framed}
\setlength{\fboxsep}{.8em}

% These colours were manually entered, they shouldn't matter unless you want pdf output

\newenvironment{redbox}{
  \definecolor{shadecolor}{RGB}{243, 154, 157}
  \color{white}
  \begin{shaded}}
 {\end{shaded}}

\newenvironment{bluebox}{
  \definecolor{shadecolor}{RGB}{172, 210, 237}
  \color{white}
  \begin{shaded}}
 {\end{shaded}}

\newenvironment{greenbox}{
  \definecolor{shadecolor}{RGB}{141, 181, 128}
  \color{white}
  \begin{shaded}}
 {\end{shaded}}
\usepackage{bookmark}
\IfFileExists{xurl.sty}{\usepackage{xurl}}{} % add URL line breaks if available
\urlstyle{same}
\hypersetup{
  pdftitle={Introductory Spatial 'Omics Analysis 2024},
  pdfauthor={Faculty: Shamini Ayyadhury, Ashleigh Willis, Farzaneh Aboualizadeh, Melanie Peralta, Trevor Pugh, Nia Hughes, and Jennifer Law},
  hidelinks,
  pdfcreator={LaTeX via pandoc}}

\title{Introductory Spatial 'Omics Analysis 2024}
\author{Faculty: Shamini Ayyadhury, Ashleigh Willis, Farzaneh Aboualizadeh, Melanie Peralta, Trevor Pugh, Nia Hughes, and Jennifer Law}
\date{July 9, 2024 - July 10, 2024}

\begin{document}
\maketitle

{
\setcounter{tocdepth}{1}
\tableofcontents
}
\part{Introduction}\label{part-introduction}

\chapter{Workshop Info}\label{workshop-info}

Welcome to the 20224 Introductory Spatial 'Omics Analysis Canadian Bioinformatics Workshop webpage!

\section{Schedule}\label{schedule}

\begin{longtable}[]{@{}
  >{\centering\arraybackslash}p{(\linewidth - 6\tabcolsep) * \real{0.1231}}
  >{\centering\arraybackslash}p{(\linewidth - 6\tabcolsep) * \real{0.4846}}
  >{\centering\arraybackslash}p{(\linewidth - 6\tabcolsep) * \real{0.1231}}
  >{\centering\arraybackslash}p{(\linewidth - 6\tabcolsep) * \real{0.2692}}@{}}
\toprule\noalign{}
\begin{minipage}[b]{\linewidth}\centering
\textbf{Time (EST)}
\end{minipage} & \begin{minipage}[b]{\linewidth}\centering
\textbf{July 9}
\end{minipage} & \begin{minipage}[b]{\linewidth}\centering
\textbf{Time (EST)}
\end{minipage} & \begin{minipage}[b]{\linewidth}\centering
\textbf{July 10}
\end{minipage} \\
\midrule\noalign{}
\endhead
\bottomrule\noalign{}
\endlastfoot
7:30 & Arrivals, Check-in, and Breakfast & 7:30 & Arrivals, Check-in, and Breakfast \\
8:30 & Welcome + introductions (Nia Hughes, Jennifer Law), AWS check & 9:00 & Module 4 \\
8:45 & Keynote (Trevor Pugh) & 10:00 & Break \\
9:15 & Module 1 & 10:30 & Module 4, cont'd \\
10:45 & Break & 12:00 & Lunch \\
11:00 & Module 1 (cont'd) & 12:45 & Module 4, cont'd \\
12:15 & Lunch & 2:45 & Break \\
1:00 & Module 2 & 3:00 & Module 5 \\
3:15 & Break & 4:00 & Break \\
3:30 & Module 3 & 4:15 & Module 6 \\
4:45 & Break & 5:00 & Concluding Remarks \\
5:00 & Module 3 cont'd & 5:30 & Finished \\
6:30 & Social & & \\
\end{longtable}

\section{Pre-work}\label{pre-work}

\href{https://docs.google.com/forms/d/e/1FAIpQLSfumnxX1DmuJ9Ve8veUzTAxzwS7eKyA8DilyO7fUoHK_U4U1w/viewform}{You can find your pre-work here.}

\chapter{Data and Compute Setup}\label{data-and-compute-setup}

\subsubsection{Course data downloads}\label{course-data-downloads}

\begin{itemize}
\tightlist
\item
  \href{https://hpc4health.ca/cbw/2024/ISO/data.tar.gz}{Data}
\item
  \href{https://hpc4health.ca/cbw/2024/ISO/images.tar.gz}{Images}
\item
  \href{https://hpc4health.ca/cbw/2024/ISO/spatial_analysis_workshop_iso2024.tar.gz}{spatial\_analysis\_workshop}
\end{itemize}

\subsubsection{Compute setup}\label{compute-setup}

We have made our AWS AMI (Amazon Machine Image) publicly available. To launch your own instance, follow the instructions provided by Amazon on \href{https://repost.aws/knowledge-center/launch-instance-custom-ami}{how to launch an EC2 instance from a custom Amazon Machine Image}. Please note that you will need an AWS account to proceed, and that you will need to upload the CourseData files yourself.

Here are the details of the AMI:

AWS Region: us-east-1 (N. Virgina)
AMI type: public image
AMI name: CBW\_ISO\_240702
AMI ID: ami-0e35aedc51010d5b4

If you want to create and activate a new AWS account, please follow the \href{https://aws.amazon.com/premiumsupport/knowledge-center/create-and-activate-aws-account/}{instructions} provided by Amazon.

\part{Modules}\label{part-modules}

\chapter{Keynote}\label{keynote}

\chapter{Module 1}\label{module-1}

\section{Lecture}\label{lecture}

\chapter{Module 2}\label{module-2}

\section{Lecture}\label{lecture-1}

\section{Lab}\label{lab}

\href{https://github.com/dr-sayyadhury/Introductory_Spatial_Analysis_Workshop___ISO_2024/tree/main/scripts/module2}{Link to scripts}

\chapter{Module 3}\label{module-3}

\section{Lecture}\label{lecture-2}

\section{Lab}\label{lab-1}

\href{https://github.com/dr-sayyadhury/Introductory_Spatial_Analysis_Workshop___ISO_2024/tree/main/scripts/module3}{Link to scripts}

\chapter{Module 4}\label{module-4}

\section{Lecture}\label{lecture-3}

\section{Lab}\label{lab-2}

\href{https://github.com/dr-sayyadhury/Introductory_Spatial_Analysis_Workshop___ISO_2024/tree/main/scripts/module4}{Link to scripts}

\chapter{Module 5/6}\label{module-56}

\section{Lecture}\label{lecture-4}

\section{Lab}\label{lab-3}

\href{https://github.com/dr-sayyadhury/Introductory_Spatial_Analysis_Workshop___ISO_2024/tree/main/scripts/module5}{Link to scripts}

\chapter{Wrap-up}\label{wrap-up}

\section{Lecture}\label{lecture-5}

\section{Lab}\label{lab-4}

\href{https://github.com/dr-sayyadhury/Introductory_Spatial_Analysis_Workshop___ISO_2024/tree/main/scripts/module6}{Link to scripts}

\bibliography{book.bib,packages.bib}

\end{document}
